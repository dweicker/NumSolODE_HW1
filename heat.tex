\subsection{Flux vector}

From the course text book we know that 
\begin{align*}
\text{flux} = - \beta q_x
\end{align*}
when the heat equation is in the form
\begin{align*}
q_t = \beta q_{xx} + S
\end{align*}
Comparing with our equation
\begin{align*}
q_t = \nabla \cdot (\nabla q) + S
\end{align*}
we can see that $\beta = 1$. Therefore the flux vector is $-\nabla q$.
\subsection{Q(t)}

We are now interested in computing : 
$$Q(t) = \int_0^1\int_0^1 q(x,y,t)dxdy$$

Using the fundamental theorem of calculus, we have :
$$Q(t) = Q(0) + \int_0^t Q'(\tau)d\tau$$

The initial condition gives :
$$Q(0) =\int_0^1\int_0^1 q(x,y,0)dxdy = \int_0^1\int_0^1 0 \: dxdy = 0$$

We also know that (using the definition of $Q$) :
$$Q'(t) = \frac{d}{dt}(\int_0^1\int_0^1 q(x,y,t)dxdy)=\int_0^1\int_0^1 q_t(x,y,t)dxdy$$

We are now going to use the PDE to substitute $q_t$. This yields :
$$Q'(t) = \int_0^1\int_0^1 q_t(x,y,t)dxdy = \int_0^1\int_0^1 \nabla \cdot (\nabla q(x,y,t))dxdy+\int_0^1\int_0^1 S(x,y,t) dxdy$$

With divergence theorem, and if $S$ means the boundary of the domain and \textbf{n} the outward unit normal to the boundary, we have : 
$$Q'(t) = \oint_S \nabla q(x,y,t)\cdot \textbf{n} \: ds + \int_0^1\int_0^1 S(x,y,t)dxdy$$

Because the given boundary condition given is :
$$\nabla q(x,y,t)\cdot \textbf{n} = 0$$

We finally have : 
$$Q'(t) = \int_0^1\int_0^1 S(x,y,t)dxdy$$

This yields an expression for $Q$ as a function of $t$.
$$Q(t) = \int_0^t\int_0^1\int_0^1 S(x,y,\tau)dxdyd\tau$$

Because $S$ is a given function, this expression can always be computed as a function of $t$.
