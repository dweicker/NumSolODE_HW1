\documentclass[11pt,a4paper]{article}

\usepackage[T1]{fontenc}
\usepackage[utf8]{inputenc}
\usepackage[english]{babel}
\usepackage{lmodern}
%\usepackage{circuitikz}
\usepackage{color}
\usepackage{wrapfig}
\usepackage{placeins}
\usepackage{subfigure}
\usepackage{tabu}
\usepackage{fullpage}
\usepackage[squaren]{SIunits}
\usepackage{graphicx}
%\usepackage[pdftex]{graphicx}
\usepackage{epstopdf}
\usepackage{epsfig}
\usepackage{hyperref}
\usepackage{tikz}
\usepackage{tikz-qtree}
\usepackage{eurosym}
%\usepackage{chemist}
\usepackage{amsmath}
\usepackage{amssymb}
\usepackage{mathrsfs}
\usepackage{dsfont}% use $\mathds{1}$
\newcommand{\C}{\mathbb{C}}
\newcommand{\N}{\mathbb{N}}
\newcommand{\Z}{\mathbb{Z}}
\newcommand{\R}{\mathbb{R}}
\newcommand{\red}{\textcolor{red}}
\newcommand{\dis}{\displaystyle}
\newcommand{\dr}{\partial}
\newcommand{\txt}{\text}
\newcommand{\td}{\todo[inline]}
\newcommand{\ttt}{\texttt}
\newcommand{\itt}{\textit}

\usepackage{algorithm}
\usepackage{todonotes}
\usepackage[noend]{algpseudocode}

%\newtheorem{theoreme}			     {Théorème}	[chapter]
%\newtheorem{proposition}[theoreme]	 {Proposition}	
%\newtheorem{corollaire}	  [theoreme]	 {Corollaire}	
%\newtheorem{lemme}	      [theoreme]  {Lemme}		
%\newtheorem{definition}	         {Définition}[chapter]
%\theoremstyle{definition}
%\newtheorem{exemple}			     {Exemple}	[chapter]
%\newtheorem{contreexemple}[exemple]{Contre-exemple}
%\newtheorem{probleme}	             {Probl\`eme}[chapter]

\usepackage{listings}
\usepackage{textcomp}
\definecolor{listinggray}{gray}{0.9}
\definecolor{lbcolor}{rgb}{0.9,0.9,0.9}
\lstset{
	backgroundcolor=\color{lbcolor},
	tabsize=4,
	rulecolor=,
	language=matlab,
        basicstyle=\scriptsize,
        upquote=true,
        aboveskip={1.5\baselineskip},
        columns=fixed,
        showstringspaces=false,
        extendedchars=true,
        breaklines=true,
        prebreak = \raisebox{0ex}[0ex][0ex]{\ensuremath{\hookleftarrow}},
        frame=single,
        showtabs=false,
        showspaces=false,
        showstringspaces=false,
        identifierstyle=\ttfamily,
        keywordstyle=\color[rgb]{0,0,1},
        commentstyle=\color[rgb]{0.133,0.545,0.133},
        stringstyle=\color[rgb]{0.627,0.126,0.941},
}

\DeclareMathOperator{\e}{e}

\title{Titre}
\author{David Weicker}
\date{\today}

\begin{document}
\tabulinesep=1.2mm
\begin{center}
\hrule
\begin{tabular}{c}
\\[0.005cm]
\Large{Numerical Solutions of Differential Equations : HW1}\\[0.3cm] %THIS IS THE TITLE
\textsc{Thomas} Garrett  \& \textsc{Weicker} David\\[0.2cm]
$\text{8}^{\text{th}}$ February 2016\\[0.2cm] %THIS IS THE DATE
\end{tabular}
\hrule
\end{center}

\section*{Introduction}
This is the introduction bitches!

\section{Conservation laws}
We are given the following Euler equations in $u=(\rho, \rho v, E)$, 
\begin{align}
	\rho_t + (\rho v)_x &= 0 \quad  \text{(Conservation of mass)} \\
	(\rho v)_t +(\rho v^2 + p)_x &= 0 \quad \text{(Conservation of momentum)} \\
	E_t + (v(E+p))_x &= 0 \quad \text{(Conservation of energy)}
\end{align}
and the variable change $u = (\rho, \rho v, E) = (u_1, u_2, u_3)$. We then have 
\begin{align*}
	f(u) = (f_1, f_2, f_3)^T = (u_2, u_2^2/u_1 + p(u_1), u_2(u_3+p(u_1))/u_1)
\end{align*}
Using the u coordinates, we can write equations (1), (2), (3) as
\begin{align*}
	u_t + f(u)_x = 0
\end{align*}
From the chain rule, we can rewrite our equation as  
\begin{align}
	u_t + \frac{\partial f}{\partial u} u_x = 0
\end{align}
We now wish to find the eigenvalues and eigenvectors of the matrix 
\begin{align*}
	\frac{\partial f}{\partial u} = \begin{pmatrix}
	\frac{\partial f_1}{\partial u_1} & \frac{\partial f_1}{\partial u_2} & \frac{\partial f_1}{\partial u_3} \\
	\frac{\partial f_2}{\partial u_1} & \frac{\partial f_2}{\partial u_2} & \frac{\partial f_2}{\partial u_3} \\
	\frac{\partial f_3}{\partial u_1} & \frac{\partial f_3}{\partial u_2} & \frac{\partial f_3}{\partial u_3}
	\end{pmatrix}
\end{align*} 
Using simple partial differentiation, we calculate 
\begin{align*}
	\frac{\partial f}{\partial u} = \begin{pmatrix}
	0 & 1 & 0 \\ 
	a & b & 0 \\ 
	c & d & e
	\end{pmatrix}
\end{align*}
with $a=-(\frac{u_2}{u_1})^2 + p'(u_1)$, \\
 $b=2(\frac{u_2}{u_1})$, \\
 $c=\frac{u_2 p'(u_1)}{u_1} - \frac{u_2(u_3 + p(u_1))}{u_1^2}$, \\
 $d = \frac{u_3 + p(u_1)}{u_1}$, \\
 and $e=\frac{u_2}{u_1}$
\\ \\
To get the eigenvalues, we look at the roots of the characteristic polynomial of this matrix 
\begin{align*}
	\lambda^3 -e \lambda^2 - b \lambda^2 + b e \lambda -a \lambda + a e = 0
\end{align*}
which gives us the three eigenvalues 
\begin{align*}
	\lambda_1 &= \frac{1}{2} (b - \sqrt{4a+b^2}) \\
	\lambda_2 &= \frac{1}{2} (b + \sqrt{4a+b^2}) \\
	\lambda_3 &= e
\end{align*}
Plugging in the actual values of $a,b,$ and $e$, we get 
\begin{align*}
	\lambda_1 &= \frac{u_2}{u_1} - \sqrt{p'(u_1)} \\
	\lambda_2 &= \frac{u_2}{u_1} + \sqrt{p'(u_1)} \\
	\lambda_3 &= \frac{u_2}{u_1}
\end{align*}
We compute the eigenvectors as follows
\begin{align*}
\begin{pmatrix}
	0 & 1 & 0 \\ 
	a & b & 0 \\ 
	c & d & e
	\end{pmatrix} \begin{pmatrix}
	u_1 \\ u_2 \\ u_3
	\end{pmatrix} = \lambda_1 \begin{pmatrix}
	u_1 \\ u_2 \\ u_3
	\end{pmatrix}
\end{align*}
which gives the system of equations
\begin{align*}
u_2 &= (\frac{u_2}{u_1} - \sqrt{p'(u_1)}) u_1 \\
au_1 + bu_2 &= (\frac{u_2}{u_1} - \sqrt{p'(u_1)}) u_2 \\
cu_1 + du_2 + eu_3 &= (\frac{u_2}{u_1} - \sqrt{p'(u_1)}) u_3
\end{align*}
We pick $u_1 = 0$, and after solving we get 
\begin{align*}
\begin{pmatrix}
u_1 \\ u_2 \\u_3
\end{pmatrix} = {pmatrix}
1 \\
 \frac{u_2}{u_1} - \sqrt{p'(u_1 )} \\
 \frac{u2 \sqrt{p'(u_1)}-(u_3+p(u_2)))}{u_1} 
\end{pmatrix}
\end{align*}
We compute the second eigenvalue similarly and get 
\begin{align*}
\begin{pmatrix}
u_1 \\ u_2 \\u_3
\end{pmatrix} = \begin{pmatrix}
1 \\
 \frac{u_2}{u_1} + \sqrt{p'(u_1 )} \\
 \frac{u2 \sqrt{p'(u_1)}+(u_3+p(u_2)))}{u_1} 
\end{pmatrix}
\end{align*}
The last eigenvalue is computed from the system of equations
\begin{align*}
\begin{pmatrix}
1 \\
 \frac{u_2}{u_1} - \sqrt{p'(u_1 )} \\
 \frac{u2 \sqrt{p'(u_1)}-(u_3+p(u_2)))}{u_1} 
\end{pmatrix}, \quad \begin{pmatrix}
1 \\
 \frac{u_2}{u_1} + \sqrt{p'(u_1 )} \\
 \frac{u2 \sqrt{p'(u_1)}+(u_3+p(u_2)))}{u_1} 
\end{pmatrix}, \quad \begin{pmatrix}
 0 \\
 0  \\
 1 \\
\end{pmatrix}
\end{align*}
\subsection{Linearisation}
%We use formal linearisation, so we first define
%\begin{align*}
%	u' = \begin{pmatrix}
%	u_1' \\
%	u_2' \\ 
%	u_3' \\
%	\end{pmatrix} = \begin{pmatrix}
%	U_1+\delta \upsilon_1 \\
%	U_2+\delta \upsilon_2 \\
%	U_3+\delta \upsilon_3 \\
%	\end{pmatrix}
%\end{align*}
%where $U_{1,2,3}$ is a solution of the (1) and $\delta$ is assumed to be small. Inserting $u'$ into (1), we get
%\begin{align*}
%	u'_t + \frac{\partial f}{\partial u'} u'_x = 0 \rightarrow \\
%	(u_1')_t + (u_2')_x = 0 \\
%	(u_2')_t+(-(\frac{u_2'}{u_1'})^2 + p'(u_1'))(u_1)_x + 2(\frac{u_2'}{u_1'})(u_2')_x = 0 \\
%	(u_3')_t+(\frac{u_2' p'(u_1)}{u_1} - \frac{u_2'(u_3' + p(u_1'))}{u_1'^2})(u_1')_x + 
%	(\frac{u_3' + p(u_1')}{u_1'})(u_2')_x + \frac{u_2'}{u_1'}(u_3') = 0
%\end{align*}
%We linearise each equation separately
%\begin{align*}
%	(u_2')_t+(-(\frac{u_2'}{u_1'})^2 + p'(u_1'))(u_1)_x + 2(\frac{u_2'}{u_1'})(u_2')_x = 0 \rightarrow \\
%	-u_2'^2(u_1)_x + u_1'^2p'(u_1')(u_1)_x + 2(u_2'u_1')(u_2')_x = 0 \rightarrow \\
%	2p'(u_1')\delta \upsilon_1 U_1 + U_2^2 (\delta \upsilon_1)_x  +  \\ + \delta \upsilon_2 ( U_1(U_2)_x +  U_1 U_2 (U_2)_x + 2 U_2 (U_1)_x) + U_2 U_1 (\delta \upsilon_2)_x + U_2^2 (U_1)_x + U_2U_1(U_2)_x + p'(u_1')U_1^2 O(\delta^2)= 0
%\end{align*}
%because $\delta$ is assumed to be small, we disregard the terms of $O(\delta^2)$. Dividing by $\delta$, we get our linear equation with variable coefficients
%\begin{align*}
%a_2 \upsilon_1 + b_2 (\upsilon_1)_x + c_2 \upsilon
%\end{align*}
We have
\begin{align}
	u_t + \frac{\partial f}{\partial u} u_x = 0
\end{align}
where $\frac{\partial f}{\partial u}$ is dependent on $u$. This equation immediately becomes linear if we fix $\frac{\partial f}{\partial u}$ at a specific point. We call it 
$u_0$, and we denote $\frac{\partial f}{\partial u}(u_0)$ as $\frac{\partial f}{\partial u}$ evaluated at this point. Thus, we have a linear equation 
\begin{align*}
u_t + \frac{\partial f}{\partial u}(u_0) u_x = 0
\end{align*}
which behaves like the nonlinear equation in a neighbourhood around $u_0$. 
\subsection{Hyperbolicity}
We know that an equation of this form is hyperbolic if $\frac{\partial f}{\partial u}$ has real eigenvalues and linearly independent eigenvectors. It can be seen easily that the eigenvalues are real as long as $p'(u_1) \geq 0$. We can also see that the eigenvectors are independent as long as either $p'(u_1) \neq 0$ or $u_3+p(u_2) \neq 0$
\subsection{Transport Equation}
We now derive conditions on $a$ and $b$ for the two-dimensional transport equation 
\begin{align*}
u_t + a(x,y)u_x + b(x,y) u_y = 0
\end{align*} to be a conservation law. 
we look at the basic form of a conservation law in two dimensions
\begin{align}
u_t + \nabla f(u) = 0
\end{align}
and we find conditions which enable us to write our transport equation in this form. 
We assume 
\begin{align*}
f = \begin{pmatrix}
a(x,y) & b(x,y)
\end{pmatrix}
\end{align*}
so plugging into $(5)$ we get
\begin{align*}
u_t + \nabla (\begin{pmatrix}
a(x,y) & b(x,y)
\end{pmatrix} u) = u_t + (a(x,y))_x u + a(x,y) u_x + (b(x,y))_y u + b(x,y) u_y = 0
\end{align*}
We now notice that if 
\begin{align*}
(a(x,y))_x = - (b(x,y))_y
\end{align*}
we get 
\begin{align*}
u_t + \nabla (\begin{pmatrix}
a(x,y) & b(x,y)
\end{pmatrix} u) = u_t + a(x,y) u_x + b(x,y) u_y = 0
\end{align*}
which is our two dimensional transportation equation. Thus,
\begin{align*}
(a(x,y))_x = - (b(x,y))_y
\end{align*} is our conservation law condition. 

\section{Heat Equation}
\subsection{Flux vector}

\subsection{Q(t)}

We are now interested in computing : 
$$Q(t) = \int_0^1\int_0^1 q(x,y,t)dxdy$$

Using the fundamental theorem of calculus, we have :
$$Q(t) = Q(0) + \int_0^t Q'(\tau)d\tau$$

The initial condition gives :
$$Q(0) =\int_0^1\int_0^1 q(x,y,0)dxdy = \int_0^1\int_0^1 0 \: dxdy = 0$$

We also know that (using the definition of $Q$) :
$$Q'(t) = \frac{d}{dt}(\int_0^1\int_0^1 q(x,y,t)dxdy)=\int_0^1\int_0^1 q_t(x,y,t)dxdy$$

We are now going to use the PDE to substitute $q_t$. This yields :
$$Q'(t) = \int_0^1\int_0^1 q_t(x,y,t)dxdy = \int_0^1\int_0^1 \nabla \cdot (\nabla q(x,y,t))dxdy+\int_0^1\int_0^1 S(x,y,t) dxdy$$

With divergence theorem, and if $S$ means the boundary of the domain and \textbf{n} the outward unit normal to the boundary, we have : 
$$Q'(t) = \oint_S \nabla q(x,y,t)\cdot \textbf{n} \: ds + \int_0^1\int_0^1 S(x,y,t)dxdy$$

Because the given boundary condition given is :
$$\nabla q(x,y,t)\cdot \textbf{n} = 0$$

We finally have : 
$$Q'(t) = \int_0^1\int_0^1 S(x,y,t)dxdy$$

This yields an expression for $Q$ as a function of $t$.
$$Q(t) = \int_0^t\int_0^1\int_0^1 S(x,y,\tau)dxdyd\tau$$

Because $S$ is a given function, this expression can always be computed as a function of $t$.


\section{Discretization and implementation}
blabla yo!

\section{Numerical results}
blabla bitches

\section{Refinements}
\subsection{Variable coefficients}
In this part, we will change the equation a little and add variable coefficients so that the equation becomes : 
$$q_t = a(y)q_{xx}+b(x)q_{yy}+S$$

Using the same principles as in section 3, we get the following :
$$\frac{1}{\Delta x \Delta y} \iint_{(i,j)} q_tdxdy = \frac{1}{\Delta x \Delta y} \iint_{(i,j)} \nabla \cdot \left(\begin{array}{c}
a(y)q_x \\ 
b(x)q_y
\end{array}\right) dxdy + \frac{1}{\Delta x \Delta y}\iint_{(i,j)}Sdxdy $$ 
$$\frac{dQ_{ij}}{dt} = \frac{1}{\Delta x \Delta y}(\int_E a(y)q_xdy + \int_N b(x)q_ydx - \int_Wa(y)q_xdy - \int_S b(x)q_ydx) + S_{ij}$$

Using finite differences to evaluate $q_x$ and $q_y$, we get : 
$$\frac{dQ_{ij}}{dt} = E\frac{aQ_{i+1,j}+bQ_{i,j}+cQ_{i-1,j}}{\Delta x^2}+F\frac{dQ_{i,j+1}+eQ_{i,j}+fQ_{i,j-1}}{\Delta y^2}+S_{ij}$$

Where $a,b,c,d,e,f$ are the same coefficients as those defined in section 3.1. $E$ and $F$ take into account the variable coefficient and are defined by : 
$$E = \frac{1}{\Delta y}\int_E a(y)dy= \frac{1}{\Delta y}\int_W a(y)dy$$
$$F = \frac{1}{\Delta x}\int_N b(x)dx= \frac{1}{\Delta x}\int_S b(x)dx$$

So, if we use the same notations as in section 3.1, the only thing we have to do is premultiply $B$ and $C$ with the corresponding matrix and after that, the implementation is unchanged. We can note that if $a(y)=b(x)=1$ then $E$ and $F$ are identity matrices and we have the same rule as previously.

In this problem, we have always used the smooth source term. We also have used the trapezoidal rule to evaluate the integral. Let us define $A1$ the $n \times n$ matrix as :
$$A1 = diag(0.5(a(0)+a(\Delta y)),0.5(a(\Delta y)+a(2\Delta y)),...,0.5(a((n-1)\Delta y)+a(1)))$$

Which is a diagonal matrix containing the different evaluations of the integral of $a$. With this matrix, we can build $B$ as :
$$B1 = (A1\otimes I(m))*B$$

Identically, we can define $A2$ as :
$$A2 = diag(0.5(b(0)+b(\Delta x)),0.5(b(\Delta x)+b(2\Delta x)),...,0.5(b((m-1)\Delta x)+b(1)))$$

And then : 
$$C1 = (I(n)\otimes A2)*C$$
And finally : 
$$M = B1 + C1$$

Once we have build $M$, the methodology is exactly the same as before.

The implementation of the method is done in $variableCoeff.m$.



\subsection{Boundary conditions}
We are now going to try and change the boundary conditions to see what happens. We have a Neuman condition on the left and right boundaries (just as before except it is no longer homogeneous). And we now have a Dirichlet condition on the upper and lower boundaries. Because in the previous grid, we had no point exactly on the boundary, we are going to shift the grid of $\frac{\Delta y}{2}$ in the y-direction. That way, we will have cells at $y=0$ and $y=1$. If we still keep $n\Delta y =1$, that means that we will now have $m(n-1)$ unknowns since the value on the upper and lower boundaries are known.






\end{document}

