\subsection{Variable coefficients}
In this part, we will change the equation a little and add variable coefficients so that the equation becomes : 
$$q_t = a(y)q_{xx}+b(x)q_{yy}+S$$

Using the same principles as in section 3, we get the following :
$$\frac{1}{\Delta x \Delta y} \iint_{(i,j)} q_tdxdy = \frac{1}{\Delta x \Delta y} \iint_{(i,j)} \nabla \cdot \left(\begin{array}{c}
a(y)q_x \\ 
b(x)q_y
\end{array}\right) dxdy + \frac{1}{\Delta x \Delta y}\iint_{(i,j)}Sdxdy $$ 
$$\frac{dQ_{ij}}{dt} = \frac{1}{\Delta x \Delta y}(\int_E a(y)q_xdy + \int_N b(x)q_ydx - \int_Wa(y)q_xdy - \int_S b(x)q_ydx) + S_{ij}$$

Using finite differences to evaluate $q_x$ and $q_y$, we get : 
$$\frac{dQ_{ij}}{dt} = E\frac{aQ_{i+1,j}+bQ_{i,j}+cQ_{i-1,j}}{\Delta x^2}+F\frac{dQ_{i,j+1}+eQ_{i,j}+fQ_{i,j-1}}{\Delta y^2}+S_{ij}$$

Where $a,b,c,d,e,f$ are the same coefficients as those defined in section 3.1. $E$ and $F$ take into account the variable coefficient and are defined by : 
$$E = \frac{1}{\Delta y}\int_E a(y)dy= \frac{1}{\Delta y}\int_W a(y)dy$$
$$F = \frac{1}{\Delta x}\int_N b(x)dx= \frac{1}{\Delta x}\int_S b(x)dx$$

So, if we use the same notations as in section 3.1, the only thing we have to do is premultiply $B$ and $C$ with the corresponding matrix and after that, the implementation is unchanged. We can note that if $a(y)=b(x)=1$ then $E$ and $F$ are identity matrices and we have the same rule as previously.

In this problem, we have always used the smooth source term. We also have used the trapezoidal rule to evaluate the integral. Let us define $A1$ the $n \times n$ matrix as :
$$A1 = diag(0.5(a(0)+a(\Delta y)),0.5(a(\Delta y)+a(2\Delta y)),...,0.5(a((n-1)\Delta y)+a(1)))$$

Which is a diagonal matrix containing the different evaluations of the integral of $a$. With this matrix, we can build $B$ as :
$$B1 = (A1\otimes I(m))*B$$

Identically, we can define $A2$ as :
$$A2 = diag(0.5(b(0)+b(\Delta x)),0.5(b(\Delta x)+b(2\Delta x)),...,0.5(b((m-1)\Delta x)+b(1)))$$

And then : 
$$C1 = (I(n)\otimes A2)*C$$
And finally : 
$$M = B1 + C1$$

Once we have build $M$, the methodology is exactly the same as before.

The implementation of the method is done in $variableCoeff.m$.



\subsection{Boundary conditions}
We are now going to try and change the boundary conditions to see what happens. We have a Neuman condition on the left and right boundaries (just as before except it is no longer homogeneous). And we now have a Dirichlet condition on the upper and lower boundaries. Because in the previous grid, we had no point exactly on the boundary, we are going to shift the grid of $\frac{\Delta y}{2}$ in the y-direction. That way, we will have center of cells at $y=0$ and $y=1$. If we still keep $n\Delta y =1$, that means that we will now have $m(n-1)$ unknowns since the value on the upper and lower boundaries are known.

We use the same methodology as before. We rewrite $Q$ as $Qvect$ and we will show that we will get a system of ODE's of the following type : 
$$\frac{dQvect}{dt} = A_xQvect+A_yQvect+b_x+b_y+S$$

We already know the definition of $S$, it is unchanged. $A_x$ and $b_x$ represents the derivative in the x-direction ($b_x$ is not zero because of the non homogeneous boundary conditions). Respectively, $A_y$ and $b_y$ take the y-derivative into account.

The x-derivative operator does not change but an additional term appears. Indeed, for a cell that is on the right boundary, we no longer no flux on the East edge but instead : 
$$\int_E q_xdy = \int_E (-1)dy = -\Delta y$$ 

In the same way, for a cell on the left boundary, we have the following West flux :
$$-\int_Wq_xdy=-\int_W(-1)dy = \Delta y$$

So, if we define $T_x$ as the following $m\times m$ matrix :
$$T_x = \frac{1}{\Delta x^2}\left(\begin{array}{ccccc}
-1 & 1 & 0 & \cdots & 0 \\ 
1 & -2 & 1 & \cdots & 0 \\ 
0& \ddots & \ddots & \ddots & 0 \\ 
0 & \cdots & 1 & -2 & 1 \\
0 & \cdots & 0 & 1 & -1
\end{array}\right) $$

We can use kronecker products to express the x-derivative :
$$A_x = I(n-1)\otimes T_x$$

Where $I(n-1)$ is the $(n-1)\times (n-1)$ identity matrix. We have to take the non homogeneous conditions into account and so :
$$b_x = ones(n-1,1) \otimes \left(\begin{array}{c}
-\frac{1}{\Delta x} \\ 
0 \\ 
\vdots \\ 
0 \\ 
\frac{1}{\Delta x}
\end{array}\right)$$

The work for the y-derivative is quite similar. But because we are only solving for the interior cells, we have "ghost cells" on the upper and lower boundary. This means that the $(n-1)\times (n-1)$ matrix $T_y$ changes a bit : 
$$T_y = \frac{1}{\Delta y^2}\left(\begin{array}{ccccc}
-2 & 1 & 0 & \cdots & 0 \\ 
1 & -2 & 1 & \cdots & 0 \\ 
0& \ddots & \ddots & \ddots & 0 \\ 
0 & \cdots & 1 & -2 & 1 \\
0 & \cdots & 0 & 1 & -2
\end{array}\right) $$

And $A_y$ becomes : 
$$A_y = T_y \otimes I(m)$$

To take the Dirichtlet conditions into account we have to use $b_y$ defined by :
$$b_y = \left(\begin{array}{ccccccccc}
q(x_1,0,t) & \cdots & q(x_m,0,t) & 0 & \cdots & 0 & q(x_1,1,t) & \cdots & q(x_m,1,t)
\end{array}\right)^T $$

Where $x_i$ means the value of $x$ in the center of cell $(i,1)$ or $(i,n-1)$.

Finally, we can define $M=A_x+A_y$ and $S_{new}=S+b_x+b_y$, we get a system of the form :
$$\frac{dQvect}{dt}=MQvect+S_{new}$$

This system is of the same form as in section 3 so we can use the same method to implement implicit Euler.




