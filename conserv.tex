We are given the following Euler equations in $u=(\rho, \rho v, E)$, 
\begin{align}
	\rho_t + (\rho v)_x &= 0 \quad  \text{(Conservation of mass)} \\
	(\rho v)_t +(\rho v^2 + p)_x &= 0 \quad \text{(Conservation of momentum)} \\
	E_t + (v(E+p))_x &= 0 \quad \text{(Conservation of energy)}
\end{align}
and the variable change $u = (\rho, \rho v, E) = (u_1, u_2, u_3)$. We then have 
\begin{align*}
	f(u) = (f_1, f_2, f_3)^T = (u_2, u_2^2/u_1 + p(u_1), u_2(u_3+p(u_1))/u_1)
\end{align*}
Using the u coordinates, we can write equations (1), (2), (3) as
\begin{align*}
	u_t + f(u)_x = 0
\end{align*}
From the chain rule, we can rewrite our equation as  
\begin{align}
	u_t + \frac{\partial f}{\partial u} u_x = 0
\end{align}
We now wish to find the eigenvalues and eigenvectors of the matrix 
\begin{align*}
	\frac{\partial f}{\partial u} = \begin{pmatrix}
	\frac{\partial f_1}{\partial u_1} & \frac{\partial f_1}{\partial u_2} & \frac{\partial f_1}{\partial u_3} \\
	\frac{\partial f_2}{\partial u_1} & \frac{\partial f_2}{\partial u_2} & \frac{\partial f_2}{\partial u_3} \\
	\frac{\partial f_3}{\partial u_1} & \frac{\partial f_3}{\partial u_2} & \frac{\partial f_3}{\partial u_3}
	\end{pmatrix}
\end{align*} 
Using simple partial differentiation, we calculate 
\begin{align*}
	\frac{\partial f}{\partial u} = \begin{pmatrix}
	0 & 1 & 0 \\ 
	a & b & 0 \\ 
	c & d & e
	\end{pmatrix}
\end{align*}
with $a=-(\frac{u_2}{u_1})^2 + p'(u_1)$, \\
 $b=2(\frac{u_2}{u_1})$, \\
 $c=\frac{u_2 p'(u_1)}{u_1} - \frac{u_2(u_3 + p(u_1))}{u_1^2}$, \\
 $d = \frac{u_3 + p(u_1)}{u_1}$, \\
 and $e=\frac{u_2}{u_1}$
\\ \\
To get the eigenvalues, we look at the roots of the characteristic polynomial of this matrix 
\begin{align*}
	\lambda^3 -e \lambda^2 - b \lambda^2 + b e \lambda -a \lambda + a e = 0
\end{align*}
which gives us the three eigenvalues 
\begin{align*}
	\lambda_1 &= \frac{1}{2} (b - \sqrt{4a+b^2}) \\
	\lambda_2 &= \frac{1}{2} (b + \sqrt{4a+b^2}) \\
	\lambda_3 &= e
\end{align*}
Plugging in the actual values of $a,b,$ and $e$, we get 
\begin{align*}
	\lambda_1 &= \frac{u_2}{u_1} - \sqrt{p'(u_1)} \\
	\lambda_2 &= \frac{u_2}{u_1} + \sqrt{p'(u_1)} \\
	\lambda_3 &= \frac{u_2}{u_1}
\end{align*}
After computing, we get the corresponding eigenvectors 
\begin{align*}
\begin{pmatrix}
1 \\
 \frac{u_2}{u_1} - \sqrt{p'(u_1 )} \\
 \frac{u2 \sqrt{p'(u_1)}-(u_3+p(u_2)))}{u_1} 
\end{pmatrix}, \quad \begin{pmatrix}
1 \\
 \frac{u_2}{u_1} + \sqrt{p'(u_1 )} \\
 \frac{u2 \sqrt{p'(u_1)}+(u_3+p(u_2)))}{u_1} 
\end{pmatrix}, \quad \begin{pmatrix}
 0 \\
 0  \\
 1 \\
\end{pmatrix}
\end{align*}
\subsection{Linearisation}
We use formal linearisation, so we first define
\begin{align*}
	u' = \begin{pmatrix}
	u_1' \\
	u_2' \\ 
	u_3' \\
	\end{pmatrix} = \begin{pmatrix}
	U_1+\delta \upsilon_1 \\
	U_2+\delta \upsilon_2 \\
	U_3+\delta \upsilon_3 \\
	\end{pmatrix}
\end{align*}
where $U_{1,2,3}$ is a solution of the (1) and $\delta$ is assumed to be small. Inserting $u'$ into (1), we get
\begin{align*}
	u'_t + \frac{\partial f}{\partial u'} u'_x = 0 \rightarrow \\
	(u_1')_t + (u_2')_x = 0 \\
	(u_2')_t+(-(\frac{u_2'}{u_1'})^2 + p'(u_1'))(u_1)_x + 2(\frac{u_2'}{u_1'})(u_2')_x = 0 \\
	(u_3')_t+(\frac{u_2' p'(u_1)}{u_1} - \frac{u_2'(u_3' + p(u_1'))}{u_1'^2})(u_1')_x + 
	(\frac{u_3' + p(u_1')}{u_1'})(u_2')_x + \frac{u_2'}{u_1'}(u_3') = 0
\end{align*}
We linearise each equation separately
\begin{align*}
	(u_2')_t+(-(\frac{u_2'}{u_1'})^2 + p'(u_1'))(u_1)_x + 2(\frac{u_2'}{u_1'})(u_2')_x = 0 \rightarrow \\
	-u_2'^2(u_1)_x + u_1'^2p'(u_1')(u_1)_x + 2(u_2'u_1')(u_2')_x = 0 \rightarrow \\
	2p'(u_1')\delta \upsilon_1 U_1 + U_2^2 (\delta \upsilon_1)_x  +  \\ + \delta \upsilon_2 ( U_1(U_2)_x +  U_1 U_2 (U_2)_x + 2 U_2 (U_1)_x) + U_2 U_1 (\delta \upsilon_2)_x + U_2^2 (U_1)_x + U_2U_1(U_2)_x + p'(u_1')U_1^2 O(\delta^2)= 0
\end{align*}
because $\delta$ is assumed to be small, we disregard the terms of $O(\delta^2)$. Dividing by $\delta$, we get our linear equation with variable coefficients
\begin{align*}
a_2 \upsilon_1 + b_2 (\upsilon_1)_x + c_2 \upsilon
\end{align*}
\subsection{Hyperbolicity}
We know that an equation of this form is hyperbolic if $\frac{\partial f}{\partial u}$ has real eigenvalues and linearly independent eigenvectors. It can be seen easily that the eigenvalues are real as long as $p'(u_1) \geq 0$. WHAT ABOUT EIGEN VECTORS? can p' = 0??